\documentclass[12pt]{letter}
\usepackage[T1]{fontenc}
\usepackage{calc}
\usepackage{setspace}
\usepackage{multicol}
\usepackage{fancyheadings}

\usepackage{graphicx}
\usepackage{color}
\usepackage{rotating}
\usepackage{harvard}
\usepackage{aer}
\usepackage{aertt}
\usepackage{verbatim}

\setlength{\voffset}{0in}
\setlength{\topmargin}{0pt}
\setlength{\hoffset}{0pt}
\setlength{\oddsidemargin}{0pt}
\setlength{\headheight}{0pt}
\setlength{\headsep}{0.5in}
\setlength{\marginparsep}{0pt}
\setlength{\marginparwidth}{0pt}
\setlength{\marginparpush}{0pt}
\setlength{\footskip}{.1in}
\setlength{\textwidth}{6.5in}
\setlength{\textheight}{8in}

\newcommand{\bi}{\begin{itemize}}
\newcommand{\ei}{\end{itemize}}
\newcommand{\be}{\begin{enumerate}}
\newcommand{\ee}{\end{enumerate}}
\newcommand{\bd}{\begin{description}}
\newcommand{\ed}{\end{description}}
\newcommand{\prbf}[1]{\textbf{#1}}
\newcommand{\prit}[1]{\textit{#1}}
\newcommand{\beq}{\begin{equation}}
\newcommand{\eeq}{\end{equation}}
\newcommand{\bdm}{\begin{displaymath}}
\newcommand{\edm}{\end{displaymath}}
\newcommand{\script}[1]{\begin{cal}#1\end{cal}}
\newcommand{\citee}[1]{\citename{#1} (\citeyear{#1})}
\newcommand{\h}[1]{\hat{#1}}
\newcommand{\ds}{\displaystyle}

\pagestyle{empty}
\begin{document}
\vspace*{-0.5in}

Dahl School of Business\\
Viterbo University\\
900 Viterbo Dr.\\
La Crosse, WI  54601

\today\\

Editorial Office\\
\textit{B.E. Journal of Macroeconomics}\\ 

To Whom It May Concern:

I am writing to request my paper, ``Regime Switching, Learning, and the Great Moderation,'' be considered for publication in the \textit{B.E. Journal of Macroeconomics}.  The purpose of the paper is to investigate how adaptive learning impacts the ``good luck vs. bad luck'' explanation for the change in macroeconomic volatility that occurred from the early 1970s through the ``Great Moderation'' period following the early 1980s.  Sims and Zha (2006) for example find statistical evidence that changes in luck, or more specifically, exogenous changes in volatility, better explain the Great Moderation than changes in monetary policy.  There is a growing adaptive learning literature within monetary economics that examines a third possibility - adaptive learning expectations mechanisms can generate time-varying volatility and possibly explain the run-up of volatility during the 1970s, and the subsequent decline of volatility after the early 1980s.  My paper allows for both possibilities - bad luck modeled similarly to Sims and Zha's work, and adaptive learning.  I find that exogenous changes in volatility are still an important factor in explaining the fit of a New Keynesian model to data on inflation, output, and interest rates, but learning does lead to lower predictions for the degree of volatility in the high volatile regime.  That is, changes in volatility are necessary, but the size of the change is smaller if assuming adaptive learning instead of rational expectations.

My paper contributes both to the literature on the causes of the Great Moderation and to the adaptive learning monetary literature that examines the impacts that learning has on the dynamics of standard monetary models, such as the New Keynesian model. 

Thank you for considering this paper for publication.  I look forward to reading the comments provided by the referees.

Sincerely,\\\\\\
James Murray
\end{document}

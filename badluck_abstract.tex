This paper examines the ``bad luck'' explanation for changing volatility in U.S. inflation and output when agents do not have rational expectations, but instead form expectations through least squares learning with an endogenously changing learning gain.  It has been suggested that this type of learning can create periods of excess volatility without the need for exogenous changes in volatility.  Bad luck is modeled into a standard New Keynesian model by augmenting it with two states that evolve according to a Markov chain, where one state is characterized by large variances for structural shocks, and the other state has relatively smaller variances.  The model is estimated by maximum likelihood, and the results show that learning does lead to lower variances for the shocks in the volatile regime, but changes in regime is still significant in explaining the change in volatility occuring at the onset of the Great Moderation.

\documentclass[12pt]{article}
\usepackage[T1]{fontenc}
\usepackage{calc}
\usepackage{setspace}
\usepackage{multicol}
\usepackage{fancyheadings}

\usepackage{graphicx}
\usepackage{color}
\usepackage{rotating}
\usepackage{harvard}
\usepackage{aer}
\usepackage{aertt}
\usepackage{verbatim}

\setlength{\voffset}{0in}
\setlength{\topmargin}{0pt}
\setlength{\hoffset}{0pt}
\setlength{\oddsidemargin}{0pt}
\setlength{\headheight}{0pt}
\setlength{\headsep}{.4in}
\setlength{\marginparsep}{0pt}
\setlength{\marginparwidth}{0pt}
\setlength{\marginparpush}{0pt}
\setlength{\footskip}{.1in}
\setlength{\textwidth}{6.5in}
\setlength{\textheight}{8.5in}
\setlength{\parskip}{0pc}
\setlength{\parindent}{0pc}

\renewcommand{\baselinestretch}{1.1}

\newcommand{\bi}{\begin{itemize}}
\newcommand{\ei}{\end{itemize}}
\newcommand{\be}{\begin{enumerate}}
\newcommand{\ee}{\end{enumerate}}
\newcommand{\bd}{\begin{description}}
\newcommand{\ed}{\end{description}}
\newcommand{\prbf}[1]{\textbf{#1}}
\newcommand{\prit}[1]{\textit{#1}}
\newcommand{\beq}{\begin{equation}}
\newcommand{\eeq}{\end{equation}}
\newcommand{\beqa}{\begin{eqnarray}}
\newcommand{\eeqa}{\end{eqnarray}}
\newcommand{\bdm}{\begin{displaymath}}
\newcommand{\edm}{\end{displaymath}}
\newcommand{\script}[1]{\begin{cal}#1\end{cal}}
\newcommand{\citee}[1]{\citename{#1} (\citeyear{#1})}
\newcommand{\h}[1]{\hat{#1}}
\newcommand{\ds}{\displaystyle}

\newcommand{\app}
{
\appendix
}

\newcommand{\appsection}[1]
{
\let\oldthesection\thesection
\renewcommand{\thesection}{Appendix \oldthesection}
\section{#1}\let\thesection\oldthesection
\renewcommand{\theequation}{\thesection\arabic{equation}}
\setcounter{equation}{0}
}

\pagestyle{fancyplain}
\lhead{}
\chead{Murray - Faculty Development Funds Request}
\rhead{\thepage}
\lfoot{}
\cfoot{}
\rfoot{}

\begin{document}
\thispagestyle{empty}
\begin{center}\textbf{Faculty Development Grants Fund Request}\end{center}

I request funding to finance the costs of participating in the 9th Annual Missouri Economics Conference, which will be taking place on March 27 and 28 at the University of Missouri - Columbia.  I will be presenting a working paper that is an extension of my dissertation work conducted at Indiana University.  The paper I will be presenting, ``Regime Switching, Learning, and the Great Moderation,'' is attached to the end of this document. \\

I presented at the Missouri Economics Conference in March 2007 and it was very beneficial to the development of my dissertation research.  The conference is co-sponsored by the University of Missouri, the Federal Reserve Bank of St. Louis, and the Federal Reserve Bank of Kansas city.  Given the sponsorship and the geographical proximity to two Federal Reserve Banks, the conference attracts many researchers in monetary economics (the specific field my work contributes to).  My previous participation in the conference was very beneficial in that I made contacts with other researchers in my discipline, and I received productive criticism for my work.  The conference also attracts some Ph.D. students working in macroeconomics (a broader discipline that includes monetary economics).  This is an exciting environment to learn about new developments in macroeconomics, and make contacts with some very energetic research economists. \\

Attached to this document are two additional documents: my current curriculum vitae, and the the paper I will be presenting.\\

Following are my responses to the seven questions requested on the Faculty Development Grant Application. \\

\textbf{A. Please provide a description of the proposed activity and how it relates to your prior work and interests.} \\

I will be presenting a working paper titled, ``Regime Switching, Learning, and the Great Moderation.''  This work is an extension of my dissertation research conducted at Indiana University.  This paper is part of my primary research program that involves investigating how people, businesses, and governments form expectations, and how these expectations mechanisms influence macroeconomic volatility, i.e. volatility in unemployment, production, inflation, and interests rates.  \\

\textbf{B. Please specify the outcomes that you hope to accomplish by engaging in the proposed activity.  For at least two outcomes, specify a plan to evaluate whether you have accomplished the outcomes.} \\

\be
\item I plan to receive productive criticism of this working paper from both young and experienced research economists in macroeconomics and monetary economics.  I have attended this same conference in the past, and received useful comments on my research that even my dissertation committee couldn't replace.  Using this advice, I plan to have the paper ready for submission to a peer reviewed scholarly publication by May 2009.  This outcome can be verified by a successful revision of the paper and submission to a scholarly journal following the conference.

\item By attending other researchers presentations and talking with other researchers in my field, I hope to gain ideas for new papers for my research program.  This outcome can be evaluated by reviewing my latest working papers.  As a member of the Social Science Research Network, my working papers can be viewed online at http://ssrn.com/author=684582.

\item I hope to learn new concepts in macroeconomics, microeconomics, and econometrics, that I can bring into my economics, statistics, and research methods classes.  As the conference sessions have not yet been posted, I do not know what papers or topics I might learn about.  I enjoy academic conferences, I have always been actively engaged with the other researchers, and I always learn something.  At the 2007 conference I attended, I learned topics as diverse as the effects of the economy on obesity, an example I bring to my microeconomics class, to topics such as the predicted length and severity that oil price shocks have on unemployment and inflation, an example I bring to my macroeconomics class.

\item It's possible that I might form friendships and professional working relationships with other research economists in my field that might lead to collaborative research work.  I am always seeking to discuss my research and other people's research, to learn more and develop as an academic economist.
\ee

\textbf{C. The Faculty Development Committee assumes the learning that is derived from faculty development activities will be integrated in some way into one's teaching.  Please describe how you hope to integrate your learning from the proposed activity into your current or future courses.} \\

As I discussed above, conferences such as this always deliver new and diverse ideas, and it is difficult to predict the impact of these ideas before knowing the full details of the papers presented at the conference.  The Missouri Economics Conference attracts a good deal of macroeconomics and econometrics research, but also attracts a good deal of research in microeconomics.  I hope to incorporate ideas I learn into any of the following classes I teach at Viterbo University:
\bi
\item Econ 101: Macroeconomics
\item Econ 102: Microeconomics
\item Math 130: Introductory Statistics
\item Mgmt 230: Managerial Statistics
\item Mgmt 560: Management and Decision Sciences
\item Mgmt 652: Integrative Research Project I
\item Mgmt 662: Integrative Research Project II
\item Mgmt 672: Integrative Research Project III
\ei

As I mentioned above, I have used knowledge I gained from this conference in 2007 in my macroeconomics and microeconomics courses.  Many of the papers presented at this conference, including my own, are empirical, and so I expect to learn about research that is relevant to the statistics classes and the research methods classes.  Furthermore, criticism I receive on my paper will likely be particularly relevant to topics I teach in macroeconomics.\\

\textbf{D. Please describe how your professional activity will contribute to your development beyond the classroom.} \\

As I discuss in the outcomes in question (B), I expect participating in this conference will allow me to develop the particular paper I am presenting so that it is ready for scholarly publication.  Through watching other researchers presentation, and talking with researchers outside the presentations, I expect to learn and discover new ideas for papers that are consistent with my research program. \\

\textbf{E. Do you have the support of your dean or department chair for this project?} \\
Yes, I have the support of Professor Tom Knothe, Dean of the Dahl School of Business. \\

\textbf{F. If you are applying for an International Grant, how will your project meet the desired outcomes of Viterbo's cross-cultural and exchange program?} \\
Not applicable.\\

\newpage

\textbf{G. Please provide a timetable and budget for your proposed activity.  Please note that the project must fall within the period deadlines established by the Faculty Development Committee.}\\

\begin{tabular}{p{1.9in}|p{1.4in}|p{1.9in}|p{0.6in}}
\textbf{Item} & \textbf{Dates} & \textbf{Details} &\textbf{Costs} \\ \hline
Driving from La Crosse, WI to Columbia, MO. & March 26, 2009. & Using own vehicle. Distance: 586 miles.  Cost per mile = \$0.44. & \$257.84. \\ \hline
Hotel stay at Stoney Creek Inn and Conference Center. & March 26-28, 2009. & Advertised cost for 3 nights: \$75/night + approximately \$15/night tax. & \$270.00. \\ \hline
Meals and Incidental Expenses Per Diem. & March 26-29, 2009. & GSA rates for Columbia, MO. = \$39/day.  First/Last day = \$29.25. & \$136.50. \\ \hline 
Driving from Columbia, MO to La Crosse, WI. & March 29, 2009. & Using own vehicle. Distance: 586 miles.  Cost per mile = \$0.44. & \$257.84. \\ \hline
\multicolumn{3}{r|}{\textbf{Total Expenses}} & \textbf{\$922.18}. \\
\end{tabular}



\end{document}



